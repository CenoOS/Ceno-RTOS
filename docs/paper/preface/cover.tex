% !Mode:: "TeX:UTF-8"

%%  可通过增加或减少 setup/format.tex中的
%%  第274行 \setlength{\@title@width}{8cm}中 8cm 这个参数来 控制封面中下划线的长度。

\cheading{天津大学~2019~届本科生毕业论文}      % 设置正文的页眉,需要填上对应的毕业年份
\ctitle{本科生毕业论文}    % 封面用论文标题,自己可手动断行
\caffil{智能与计算学部} % 学院名称
\csubject{软件工程}   % 专业名称
\cgrade{2015~级}            % 年级
\cauthor{杨兴锋}            % 学生姓名
\cnumber{3015218102}        % 学生学号
\csupervisor{毕重科}        % 导师姓名
% \crank{副教授}              % 导师职称

\cdate{\the\year~年~\the\month~月~\the\day~日}

\cabstract{
嵌入式计算机系统正在激增,
但嵌入式软件的复杂性使得生产稳健可靠的系统变得越来越困难。
随着嵌入式系统连接到网络并依赖于它控制或监控关键基础设施中的物理过程,
这些挑战也在增加。
本文描述了一个运行于TivaC-LaunchPadGXL123 (ARM Cortex-M4 MCU TM4C123GH6PM)\cite{morales2013introduction}上的RTOS的实现,
向读者展示了一个RTOS实现所需要的必要知识和该RTOS实现的所有细节,
希望可以有效地帮助读者获得一些嵌入式软件开发人员所需的知识和技能。
}

\ckeywords{~RTOS;~CORTEX-M;~ARM;~多线程;~任务调度}

\eabstract{
Embedded computer systems are proliferating, but the complexities of embedded software make it increasingly difficult to produce systems that are robust and reliable. These challenges increase as embedded systems are connected to networks and relied on to control or monitor physical processes in critical infrastructure.
This article describes an implementation of an RTOS running on TivaC-LaunchPadGXL123 (ARM Cortex-M4 MCU TM4C123GH6PM).
Explain to the reader the necessary knowledge of an RTOS implementation and all the details of the RTOS implementation.
I hope to effectively help readers acquire the knowledge and skills required by some embedded software developers. 
}

\ekeywords{~RTOS;~CORTEX-M;~ARM;~Multithreading;~Scheduing}

\makecover

\clearpage
